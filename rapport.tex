\documentclass[11pt,a4paper]{article}



\usepackage{fontspec}
\setmainfont[Ligatures=TeX]{TeX Gyre Pagella}
\setmonofont{DejaVu Sans Mono}
%\usepackage[frenchb]{babel} % Global stuff set to french
\usepackage[margin=2cm]{geometry} % The margin of the page
%\usepackage{amsmath}  % to include math formulas
\usepackage{graphicx} % to include pictures
\usepackage[hidelinks]{hyperref} % To include hyperlinks in a PDF
\usepackage{fancyhdr} % to be able to make the page fancy looking
\usepackage{lastpage} % so latex knows what is the last page...
\usepackage{color} % For text colors
%\usepackage{tabularx}
\usepackage{listings}

%% Fancy layout
\pagestyle{fancy}
    \lhead{INFO-F201 - Projet 2}
    \chead{}
    \rhead{Carlos Requena - \emph{410031}}
    \lfoot{}
    \cfoot{}
    \rfoot{Page \thepage\ de \pageref{LastPage}}
\renewcommand{\headrulewidth}{0.4pt}
\renewcommand{\footrulewidth}{0.4pt}

\definecolor{mygreen}{rgb}{0,0.6,0}
\definecolor{mygray}{rgb}{0.41,0.41,0.41}
\definecolor{mymauve}{rgb}{0.85,0,0}
\definecolor{myblue}{rgb}{0, 0.2, 0.9}
\definecolor{mybackground}{RGB}{245, 245, 245}


\lstset{ %
  backgroundcolor=\color{mybackground},   % choose the background color; you must add \usepackage{color} or \usepackage{xcolor}
  basicstyle=\normalsize\ttfamily,        % the size of the fonts that are used for the code
  breakatwhitespace=false,         % sets if automatic breaks should only happen at whitespace
  breaklines=true,                 % sets automatic line breaking
  captionpos=b,                    % sets the caption-position to bottom
  commentstyle=\color{mygreen},    % comment style
  columns=flexible,
  deletekeywords={...},            % if you want to delete keywords from the given language
  escapeinside={\%*}{*)},          % if you want to add LaTeX within your code
  extendedchars=true,              % lets you use non-ASCII characters; for 8-bits encodings only, does not work with UTF-8
  frame=trBL,                    % adds a frame around the code
  keepspaces=true,                 % keeps spaces in text, useful for keeping indentation of code (possibly needs columns=flexible)
  keywordstyle=\color{myblue},       % keyword style
  language=c,                 % the language of the code
  morekeywords={*,...},            % if you want to add more keywords
                                % to the set
  inputencoding=utf8,
  numbers=left,                    % where to put the line-numbers; possible values are (none, left, right)
  numbersep=9pt,                   % how far the line-numbers are from the code
  numberstyle=\footnotesize\color{mygray}, % the style that is used for the line-numbers
  rulecolor=\color{black},         % if not set, the frame-color may be changed on line-breaks within not-black text (e.g. comments (green here))
  %showspaces=false,                % show spaces everywhere adding particular underscores; it overrides 'showstringspaces'
  %showstringspaces=false,          % underline spaces within strings only
  showtabs=false,                  % show tabs within strings adding particular underscores
  stepnumber=2,                    % the step between two line-numbers. If it's 1, each line will be numbered
  stringstyle=\color{mymauve},     % string literal style
  title=\lstname                   % show the filename of files included with \lstinputlisting; also try caption instead of title
}


%%% --- %%% --- DOCUMENT START --- %%% --- %%%
\begin{document}
\pagestyle{fancy}

\section{Introduction}
\label{sec:intro}

Le but de ce projet est la réalisation d'une application
Client-Serveur mettant en place un service de jeu ``Morpion''. Les
notions de IPC (Inter Process Communication) et System Calls sont
explorés.

\section{Description d'architecture}
\label{sec:des}

Le programme est écrit en C, en utilisant tous les appels systèmes qui
permettent de communiquer avec l'API (Application Programming
Interface) des sockets. Un socket est une interface Un grand nombre de systèmes d'exploitation
offrent cette interface.

\medbreak

Ces sockets font possible la communication inter processus aussi bien
sur une même machine qu'à travers un réseau.

\medbreak

Le port!!!

Un serveur devra offrir la possibilité de connexion à un nombre
non-limité de clients. Ces derniers, se limittent à envoyer des rêquetes.

Un stream socket typique est choisie, etc...

\subsection{Gestion d'erreurs et exceptions}
\label{sec:err}

perror

\section{System calls}
\label{sec:calls}

\begin{itemize}
\item \texttt{\textbf{getaddrinfo():}} L'usage de cette fonction est optionel,
  vu que client et serveur sont dans la mâchine et on connaît le
  protocole, le type de socket et la version IP à utiliser. Néanmoins,
  ça peut servir pour étendre les fonctionnalités du programme.
\item \texttt{\textbf{socket():}} Crée un point final de communication et
  renvoie un descripteur de fichier. Les sockets peuvent avoir de
  différents types, qui correspondent à la famille de protocoles à
  utiliser.
\item \texttt{\textbf{bind():}} Fournie un nom (adresse) à un socket, à l'aide
  des structures de données précédemment completées (comme
  \texttt{sockaddr}). Avant cet appel, le socket créé avec
  \texttt{socket()} n'a pas de nom formel.
\item \texttt{\textbf{listen():}} fd
\item \texttt{\textbf{connect():}} Pour les sockets en mode connecté
  (\texttt{SOCK\_STREAM} par example, qui utilise le protocole TCP),
  cet appel extrait la première connexion de la file d'attente et
  renvoie un nouveau ``file descriptor'' qui fait référence à un
  nouveau socket. Ce descripteur de fichier peut être utilisé pour
  transmettre information entre les processus, tandis que le socket
  originel est laissé intact et continue à l'écoute.
\item \texttt{\textbf{accept():}} asdf
\item \texttt{\textbf{send():}} Permet de transmettre un message à destination
  d'une autre socket. L'appel système \texttt{write()} est le même
  sans avoir la possibilité d'ajouter des \emph{flags}.
\end{itemize}


\subsection{Structures de données}
\label{sec:struct}

Deux structures de donnees doivent être remplies (à la main ou avec
les méthodes auxilaires).

\texttt{sockaddr\_storage} contient l'addresse.

\subsection{Procotol??}
\label{sec:prot}

TCP stream socket?



\newpage

\section{Listing}

\lstinputlisting[]{client.c}
\newpage
\lstinputlisting[]{server.c}
\newpage
\lstinputlisting[language=make]{Makefile}

\end{document}
